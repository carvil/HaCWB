\section{Software Utilizado}
Aqui vamos enunciar as três principais ferramentas de software utilizadas no desenvolvimento do trabalho. Na última secção falaremos resumindamente de outras ferramentas que nos ajudaram a levar o trabalho a ``bom porto''.  \\
Como compilador principal da linguagem \haskell, utilizamos a ferramenta \textsf{GHC}. \\
Na construção do interpretador de ficheiros utilizamos a ferramenta \textsf{Alex} como \textit{lexer} e a ferramenta \textsf{Happy} como gerador do \textit{parser}.\\
O \textsf{Alex} gera a lista de \textit{tokens} reconhecidos no texto de entrada e o \textsf{Happy} interpreta esses \textit{tokens} de acordo com a linguagem especificada.

\subsection{GHC}
A ferramenta \textsf{GHC} foi escolhida para este trabalho pois ela pemite-nos compilar o trabalho e não apenas interpreta-lo. Isto é tanto ao mais importante pois é nosso objectivo gerar uma ferramenta autónoma e não apenas desenvolver um conjunto de bibliotecas para serem posteriormente interpretadas num qualquer interpretador de \haskell.\\
Assim sendo, na fase de desenvolvimento do trabalho utilizamos o interpretador \textsf{GHCi}, o que nos permitiu ir monitorizando e corrigindo os erros e quando quisemos gerar a ferramenta \textbf{hacwb} usamos o compilador \textsf{GHC}.\\
Mais informações em \url{http://www.haskell.org/ghc/}.

\subsection{Alex}
A ferramenta \textsf{Alex} pretende ser um gerador de analisadores léxicos para a linguagem {\haskell} em que cada \textit{token} é descrito sobe a forma de expressão regular. A melhor maneira de a descrever é dizer que \textsf{Alex} está para o {\haskell} como o \textsf{Flex} está para o \texttt{C/C++}.\\
Mais informações em \url{http://www.haskell.org/alex/}.

\subsection{Happy}
\textsf{Happy} pretende ser gerador de \textit{parsers} também para a linguagem \haskell, ou seja, dada uma determinada linguagem na forma \textbf{BNF} (Backus–Naur form), o \textsf{Happy} constroi o respectivo \textit{parser}. A melhor maneira para o descrever é dizer que o \textsf{Happy} está para o {\haskell} como o \textsf{YACC/Bison} está para o \texttt{C/C++}.\\
De seguida apresenta-se a linguagem utilizada no para este trabalho e as respectivas acções semânticas.
\newpage
\begin{verbatim}
Root :: { LstProcesses String }
Root : PROCNAME '=' Process ';' Root  { ($1, $3):$5 }
     | {- empty -} { [] }

Process :: {  Process String }
Process : Rests Proc_Body { if $1==[] then $2 else (New $1 $2) }
     | {- empty -} { Zero }

Rests :: { [String] }
Rests : NEW '{' More_Rests '}' { $3 }
      | {- empty -} { [] }
More_Rests : ACT ',' More_Rests { $1:$3 }
           | ACT { [$1] }

Proc_Body :: { Process String }
Proc_Body : Pro '+' Pro { Or $1 $3 }
        | Pro '|' Pro { Paralel $1 $3 } 
        | Pro { $1 }
Pro : Pro_Complex { $1 }
    | P { $1 }
Pro_Complex : '(' Proc_Body ')'{ $2 }
P : ACT '.' P { Dot (Var $1) $3 }
  | '~' ACT '.' P { Dot (Comp $2) $4 }
  | '0' { Zero }
  | PROCNAME {NameP $1}
\end{verbatim}

De notar que a cada produção da gramática está associada a respectiva acção semântica. Como se pode ver acima, o resultado final é a lista de processos contidos no ficheiro de entrada, no tipo de dados que utilizamos no nosso trabalho.\\
Mais informações em \url{http://www.haskell.org/happy/}.

\subsection{Outras}
Outra ferramenta que utilizamos no nosso trabalho foi a \textsf{Make} que vem com todas as distribuições \texttt{Unix}, permitindo desta forma automatizar o processo de criação da ferramenta. Para isso basta apenas fazer ``make hacwb'' e o executável é gerado automáticamente.
