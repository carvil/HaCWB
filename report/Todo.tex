\section{TODO}

Para que o Haskell Concurrency Workbench esteja finalizado, é necessário ainda desenvolver alguns pontos. 
Segue-se uma lista resumida do trabalho futuro proposto:
\begin{itemize}
\item Melhoramento a nível do código - Trata-se de um objectivo a curto prazo. Pretende-se melhorar o código criado, 
de forma a tornar os algoritmos mais eficientes de forma mais elegante.
\item Cálculo de Processos - trata-se da verdadeira essência do HaCWB. A introdução deste \textit{item} 
vai permitir a interacção de processos (criação de diagramas de sincronização) e ainda todo o tipo de verificações:
\begin{itemize}
\item Bissimulação e Equivalência Estrita;
\item Equivalência Observacional;
\item Igualdade de Processos.
\end{itemize}
\item Como já foi referido, será criado um \textit{parser} para \textit{$\Pi-Calculus$}, juntando-se assim ao 
parser de CCS.
\item Por fim, para facilitar a utilização desta ferramenta, será incluído um modo gráfico a ser construído utilizando 
a biblioteca \textit{WX-Haskell}.
\end{itemize}
