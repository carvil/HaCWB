
%%% Local Variables: 
%%% mode: latex
%%% TeX-master: "main"
%%% End: 

\section{Objectivos}
Nesta secção vamos falar um pouco sobre a razão que nos levou a lançar este projecto também iremos abordar mais detelhadamente  os objectivos que temos para o nosso trabalho.\\
A ideia deste pojecto, surgiu-nos quando foi propos-to na disciplina de MP4 a utilização de uma excelente ferramenta (\textsf{CWB}) que visa a animação de processos e a formulação de vários tipo de provas sobre os mesmos. Mas apesar de ser uma boa ferramenta, tem o grande defeito de ser muito difícil (para não dizer impossível) de instalar. Para ajudar ainda mais, existe na Universidade do Minho uma grande cultura para a utilização da poderosa linguagem de programação \haskell, e foi então que nos surgiu a ideia da construção de uma ferramenta em \haskell permitindo assim tirar todas as vantagens desta linguagem de programação.\\
Passando agora mais concretamente aos nossos objectivos que temos para este projecto, é nosso ``sonho'' que a nossa ferramenta tenha pelo menos as mesmas funcionalidades do \textsf{CWB}, com uma interface um pouco mais polida e ainda adicionar mais funcionalidade ao nível da interação do utilizador com o resultado da computação realizada sobre a lista de processos inicial.\\
Actualmente a ferramenta lê de um ficheiro, um conjunto de processos inicial e depois de seleccionado oo processo pelo qual deseja iniciar, é gerado o grafo de transições desse processo para um limite desejado, sendo de seguida esse resultado exportado para um ficheiro \texttt{GraphViz}. O  \texttt{GraphViz} foi escolhido como forma de mostrar o resultado pois é muito simples a geração da representação do grafo de transições de precessos arbitrariamente complexos, sendo também muito fácil a visualização por parte do utilizador do respectivo resultado.\\
Como objectivos futuros, ou seja, funcionalidades para próximas versões da ferramenta, estamos empenhados em adicionar o reconhecimento da linguagem de {\textsf{CCS}} com anotação bem como a possibilidade da realização de tarefas bem mais interessantes sobre os processos como sejam a prova de simulações, bi-simulações, equivalências observacionais, etc.

% FIM